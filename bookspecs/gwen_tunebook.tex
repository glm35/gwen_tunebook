
\documentclass[a4paper,12pt]{article}

\usepackage[frenchb]{babel}
\usepackage[T1]{fontenc}
%\usepackage[latin1]{inputenc}
%\usepackage{ucs}
\usepackage[utf8]{inputenc}

\usepackage{syntonly}
%\syntaxonly % A commenter pour générer réellement le document

\usepackage{gchords}


\author{Gwenaël \bsc{Lambrouin}}
\title{Airs irlandais}
%\date{04 janvier 2003}
% Avec le makefile, le TuneBook n'est regénéré que quand il y a une
% modification. Donc on peut bien utiliser la commande \today
\date{\today}


\begin{document}

\maketitle

% Quelques accords communs

% quelques accords pour gchords...

% nommenclatures (triturée parce que ce n'est pas possible de mettre des
% chiffres dans les noms de commandes LaTeX.
% p : position
% v : variante + lettre de numéro de variante (a, b, c, ...)
% s : sharp; f : flat
% b : bass

% Les accords de base pour jouer en Sol
\newcommand{\GMaj}{\chord{3}{p3,p3,n,n,p1,p1}{G}}
\newcommand{\Amin}{\chord{t}{x,n,2,n,1,x}{Am}}
\newcommand{\AminpV}{\chord{5}{3,x,3,1,1,1}{Am}}
\newcommand{\BminpVII}{\chord{7}{p3,x,p3,p1,p1,p1}{Bm}}
\newcommand{\CMaj}{\chord{t}{x,3,2,n,1,x}{C}}
\newcommand{\DinG}{\chord{3}{x,3,2,n,1,x}{D}}
\newcommand{\Emin}{\chord{t}{2,2,2,n,n,n}{Em}}
\newcommand{\DbFs}{\chord{t}{4,x,n,2,3,x}{D/F\#}}
% Autres accords pour jouer en Sol:
\newcommand{\Dva}{\chord{t}{n,n,4,2,3,x}{D}}
\newcommand{\EminpV}{\chord{5}{x,3,1,n,n,n}{Em}}
\newcommand{\DpV}{\chord{5}{x,1,n,3,3,x}{D}}
\newcommand{\GpVII}{\chord{7}{x,4,3,1,2,x}{G}}
\newcommand{\Gva}{\chord{3}{3,x,3,2,n,x}{G}}
\newcommand{\DpX}{\chord{10}{3,x,n,2,1,x}{D}}
\newcommand{\CpVIII}{\chord{8}{3,x,3,n,1,x}{C}}
\newcommand{\Gvb}{\chord{3}{3,x,3,2,1,x}{G}}

% Les accords de base pour jouer en La
\newcommand{\AMaj}{\chord{5}{x,n,3,2,1,n}{A}}
\newcommand{\DMaj}{\chord{t}{n,5,n,2,3,x}{D}}
\newcommand{\EMaj}{\chord{t}{2,2,2,1,n,n}{E}}
\newcommand{\Fsmin}{\chord{t}{4,x,4,2,2,2}{F\#m}}
\newcommand{\Bmin}{\chord{t}{x,2,4,4,3,2}{Bm}}

% Accords pour jouer en La Mixo
\newcommand{\Anotd}{\chord{7}{x,n,1,3,4,x}{A (no 3rd)}} % A without 3d
\newcommand{\GbA}{\chord{3}{x,n,3,2,1,x}{G/A}}
\newcommand{\GbAva}{\chord{7}{x,n,3,1,2,x}{G/A}}
\newcommand{\AMajva}{\chord{5}{x,n,3,2,1,1}{A}}
\newcommand{\DbA}{\chord{7}{x,n,1,1,1,x}{D/A}}
\newcommand{\FbA}{\chord{5}{x,n,3,1,2,x}{F/A}}


% Accords pour jouer en La mineur
\newcommand{\Aminva}{\chord{5}{x,n,3,1,1,1}{Am}}
\newcommand{\EmbA}{\chord{3}{x,n,3,2,3,x}{Em/A}}
\newcommand{\GbAvb}{\chord{5}{x,n,1,n,n,x}{G/A}}
\newcommand{\Gvc}{\chord{5}{1,x,1,n,n,x}{G}}

%\section{Introduction}

Une compilation des airs de musique irlandaise que j'ai appris à la
flûte, et (ou) les airs pour lesquels j'ai cherché un accompagnement à la
guitare.

L'intérêt est de pouvoir réviser régulièrement les morceaux, et de retrouver
rapidement un air ``oublié''.


%%INSERT_TUNES


\pagebreak
\tableofcontents


%%INSERT_INDEX

\end{document}
