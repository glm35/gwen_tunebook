\documentclass[a4paper,12pt]{article}

\usepackage[frenchb]{babel}
\usepackage[T1]{fontenc}
\usepackage[latin1]{inputenc}

\usepackage{syntonly}
%\syntaxonly % A commenter pour générer réellement le document

\usepackage{gchords}


%\author{G.~Lambrouin}
\title{Tableau d'accords en Dropped D}
\date{08 Février 2005}


\begin{document}

\maketitle

% Fichier avec la définition des accords
\input{../src/Chords}

\section*{Accords pour jouer en Ré}

\section*{Accords pour jouer en Sol}

%% \hbox{$\backslash$GMaj $\backslash$Amin \\
%% \GMaj \Amin \\}

%% \hbox{\GMaj \Amin}

\begin{tabular}{|c|c|c|c|c|c|c|c|}
\hline
$\backslash$GMaj & $\backslash$Amin & $\backslash$AminpV & $\backslash$BminpVII & $\backslash$CMaj
& $\backslash$DinG & $\backslash$Emin & $\backslash$DbFs \\
\GMaj & \Amin & \AminpV & \BminpVII & \CMaj & \DinG & \Emin & \DbFs \\

\hline
$\backslash$Dva & $\backslash$EminpV & $\backslash$DpV & $\backslash$GpVII &
$\backslash$Gva & $\backslash$Gvb & $\backslash$DpX & $\backslash$CpVIII \\
\Dva & \EminpV & \DpV & \GpVII & \Gva & \Gvb & \DpX & \CpVIII \\
\hline
\end{tabular}

\section*{Accords pour jouer en La}

\begin{tabular}{|c|c|c|c|c|}
\hline
$\backslash$AMaj & $\backslash$DMaj & $\backslash$EMaj & $\backslash$Fsmin & $\backslash$Bmin \\
\AMaj & \DMaj & \EMaj & \Fsmin & \Bmin \\
\hline
\end{tabular}

\section*{Accords pour jouer en La mixolydien}

\begin{tabular}{|c|c|c|c|c|}
\hline
$\backslash$Anotd & $\backslash$GbA & $\backslash$GbAva & $\backslash$AMajva &
$\backslash$EmbA \\
\Anotd & \GbA & \GbAva & \AMajva & \EmbA \\
\hline
\end{tabular}

\section*{Accords pour jouer en La mineur}

\begin{tabular}{|c|c|c|c|}
\hline
$\backslash$Aminva & $\backslash$GbAvb & $\backslash$Gvc \\
\Aminva & \GbAvb & \Gvc \\
\hline
\end{tabular}

\end{document}



% TODO:
% - utiliser le même espacement que la macro \chords de gchords (hbox)